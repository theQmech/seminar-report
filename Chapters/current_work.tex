
\chapter{Current work in this field}

\section{Ezyang's post summary}

Why should this section be in current work? This is already encompassed in the
exmaples section. We aren't describing a method to fix leaks over here.

Here we will be talking particularly about the blog post "Anatomy of a thunk 
leak" by ezyang(replace by aprop references).

\begin{lstlisting}
main = evaluate (f [1..4000000] (0 :: Int))

f []     c = c
f (x:xs) c = f xs (c + 1)
\end{lstlisting}
Its evident that the $c+1$, cause accumulation of closures and hence leads to
space leak. What is surprising is that compiling with optimiztions is enough to 
fix the leak(thanks to GHC). This is due to the fact that the compiler decides 
to rather use $GHC.Prim.Int\#$ - a primitive version of Int data type which 
doesn't form any closures.

The following is the core code generated with no optimizations.
\begin{lstlisting}
f_aue [Occ=LoopBreaker] :: [Integer] -> Int -> Int
[LclId, Arity=2, Str=DmdType]
f_aue =
 \ (ds_dMW :: [Integer]) (c_atb :: Int) ->
   case ds_dMW of _ [Occ=Dead] {
     [] -> c_atb;
     : x_atc xs_atd ->
       f_aue
         xs_atd
         (+ @ Int
            GHC.Num.$fNumInt
            c_atb
            (fromInteger @ Int GHC.Num.$fNumInt (__integer 1)))
   };
\end{lstlisting}

And here is the optimized leak free code using the primitive integer - 
$GHC.Prim.Int\#$

\begin{lstlisting}
Main.$wf =
  \ (w_s2PM :: [Integer]) (ww_s2PQ :: GHC.Prim.Int#) ->
    case w_s2PM of _ [Occ=Dead] {
      [] -> ww_s2PQ;
      : x_auC xs_auD -> Main.$wf xs_auD (GHC.Prim.+# ww_s2PQ 1)
    }
\end{lstlisting}

Next, we consider a slightly more complicated example

\begin{lstlisting}
import Control.Exception.Base

main = do
    evaluate (f [1..4000000] (0 :: Int, 1 :: Int))

f []     c = c
f (x:xs) c = f xs (tick x c)

tick x (c0, c1) | even x    = (c0, c1 + 1)
                | otherwise = (c0 + 1, c1)
\end{lstlisting}

This piece of code leaks even with optimizations. Unlike in the previous case, 
using $GHS.Prim.Int\#$ doesn't work because the tuple constructor is still 
lazily evaluated. The bang pattern(strictness) just enforces evaluation to WHNF
and not any further.

Insert core language code here, Suresh should be having them(mostly).

insert profiles here\\

% discuss what happens on adding bang to only c in definition of f\\
Banging the $c$ in the definiton for $f$, simply doesn't plug the leak[caption]

% discuss the standard solutions of banging patterns to c0 and c1\\
Here is one solution: bang down the $c0$ and $c1$ in definition for $f$.
Note that this works only with optimizations enabled. This is due to the fact 
that GHC has unboxed the tuple and inlined the $tick$(see core language code below). 

core language code\\

A complete solution, which works even without optimizations would be to make both $f$ anf $tick$ 
strict.

insert profiles\\

% Ezyang's post on "Calling all space leaks" also contains a lot of examples. Add them up. \\

\section{Neil Mitchell's blog posts}

Neil Mitchell made the following observation (quoting his blog) - "Every large 
Haskell program almost inevitably contains space leaks". Over a series of blog 
posts he has reported a lot of space leaks, how he found them and how he fixed
them. We present some of them over here.

\begin{lstlisting}
indexInto :: Eq a => Int -> a -> [a] -> Maybe Int
indexInto _ _ []                 = Nothing
indexInto i x (y:ys) | x == y    = Just i
                     | otherwise = indexInto (i+1) x ys
\end{lstlisting}
This peice of returns the index of occurence of an element in a list. This 
funtion should ideally use $O(1)$ space, however this implementation ends up 
using space of the order of index of occurence. This is due to the accumulating
chain of $+1$'s. 

Solution: 
\begin{lstlisting}
let j = i + 1 in j `seq` indexInto j x ys
\end{lstlisting}


\begin{lstlisting}
foldr (\(a,b) (c,d) -> (a+c,b+d)) (0,0) conflictList
\end{lstlisting}
Keeping in mind the semantics of $foldr$ in Haskell, its easy to convince 
yourself that $foldr$ is rather a bad choice and almost always leads to space 
eaks. However, in this case, even $foldl$ leads to space leaks. Not only that, 
$foldl'$, the strict version of $foldl$ forces evaluation of only the tuple 
constructor. Even this implementation is not free of the accumulating addition
closures.

Solution: 
\begin{lstlisting}
foldl' (\(a,b) (c,d) ->
    let ac = a + c; bd = b + d
    in ac `seq` bd `seq` (ac,bd))
    (0,0) conflictList
\end{lstlisting}
Or:
\begin{lstlisting}
let strict2 f !x !y = f x y
in foldr (\(a,b) (c,d) -> strict2 (,) (a+b) (b+d)) (0,0) conflictList
\end{lstlisting}

==> Some monads related optimizations, will come back to this later.




Breifly discuss the space leaks fixed by Neil Mitchell.

Then discuss his proposed trick to fixing leaks with a good example.

